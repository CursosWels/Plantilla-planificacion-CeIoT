\documentclass[
11pt, % The default document font size, options: 10pt, 11pt, 12pt
%codirector, % Uncomment to add a codirector to the title page
]{charter} 


% El títulos de la memoria, se usa en la carátula y se puede usar el cualquier lugar del documento con el comando \ttitle
\titulo{Mejora del Sistema de Apunte Automático mediante la integración de IoT para monitoreo, control y gestión remota} 

% Nombre del posgrado, se usa en la carátula y se puede usar el cualquier lugar del documento con el comando \degreename
%\posgrado{Carrera de Especialización en Sistemas Embebidos} 
\posgrado{Carrera de Especialización en Internet de las Cosas} 
%\posgrado{Carrera de Especialización en Inteligencia Artificial}
%\posgrado{Maestría en Sistemas Embebidos} 
%\posgrado{Maestría en Internet de las cosas}

% Tu nombre, se puede usar el cualquier lugar del documento con el comando \authorname
% IMPORTANTE: no omitir titulaciones ni tildación en los nombres, también se recomienda escribir los nombres completos (tal cual los tienen en su documento)
\autor{Esp. Ing. William's Ernesto Limonchi Sandoval}

% El nombre del director y co-director, se puede usar el cualquier lugar del documento con el comando \supname y \cosupname y \pertesupname y \pertecosupname
\director{Mg. Ing. Juan Carlos Espinoza Guerra}
\pertenenciaDirector{Radio Observatorio de Jicamarca} 
\codirector{} % para que aparezca en la portada se debe descomentar la opción codirector en los parámetros de documentclass
\pertenenciaCoDirector{FIUBA}

% Nombre del cliente, quien va a aprobar los resultados del proyecto, se puede usar con el comando \clientename y \empclientename
\cliente{Dr. Danny Scipión}
\empresaCliente{Radio Observatorio de Jicamarca - IGP}
 
\fechaINICIO{20 de mayo de 2025}		%Fecha de inicio de la cursada de GdP \fechaInicioName
\fechaFINALPlan{18 de junio de 2025} 	%Fecha de final de cursada de GdP
\fechaFINALTrabajo{30 de octubre de 2025}	%Fecha de defensa pública del trabajo final


\begin{document}

\maketitle
\thispagestyle{empty}
\pagebreak


\thispagestyle{empty}
{\setlength{\parskip}{0pt}
\tableofcontents{}
}
\pagebreak


\section*{Registros de cambios}
\label{sec:registro}


\begin{table}[ht]
\label{tab:registro}
\centering
\begin{tabularx}{\linewidth}{@{}|c|X|c|@{}}
\hline
\rowcolor[HTML]{C0C0C0} 
Revisión & \multicolumn{1}{c|}{\cellcolor[HTML]{C0C0C0}Detalles de los cambios realizados} & Fecha      \\ \hline
0      & Creación del documento                                 &\fechaInicioName \\ \hline
%1      & Se completa hasta el punto 5 inclusive                & {día} de {mes} de 202X \\ \hline
%2      & Se completa hasta el punto 9 inclusive
%		  Se puede agregar algo más \newline
%		  En distintas líneas \newline
%		  Así                                                    & {día} de {mes} de 202X \\ \hline
%3      & Se completa hasta el punto 12 inclusive                & {día} de {mes} de 202X \\ \hline
%4      & Se completa el plan	                                 & {día} de {mes} de 202X \\ \hline

% Si hay más correcciones pasada la versión 4 también se deben especificar acá

\end{tabularx}
\end{table}

\pagebreak



\section*{Acta de constitución del proyecto}
\label{sec:acta}

\begin{flushright}
Buenos Aires, \fechaInicioName
\end{flushright}

\vspace{2cm}

Por medio de la presente se acuerda con el \authorname\hspace{1px} que su Trabajo Final de la \degreename\hspace{1px} se titulará ``\ttitle'' y consistirá en la implementar una mejora en el sistema actual que se tiene en el Radio Observatorio de Jicamarca. El trabajo tendrá un presupuesto preliminar estimado de 600 horas y un costo estimado de \$300, con fecha de inicio el \fechaInicioName\hspace{1px} y fecha de presentación pública el \fechaFinalName.

Se adjunta a esta acta la planificación inicial.

\vfill

% Esta parte se construye sola con la información que hayan cargado en el preámbulo del documento y no debe modificarla
\begin{table}[ht]
\centering
\begin{tabular}{ccc}
\begin{tabular}[c]{@{}c@{}}Dr. Ing. Ariel Lutenberg \\ Director posgrado FIUBA\end{tabular} & \hspace{2cm} & \begin{tabular}[c]{@{}c@{}}\clientename \\ \empclientename \end{tabular} \vspace{2.5cm} \\ 
\multicolumn{3}{c}{\begin{tabular}[c]{@{}c@{}} \supname \\ Director del Trabajo Final\end{tabular}} \vspace{2.5cm} \\
\end{tabular}
\end{table}




\section{1. Descripción técnica-conceptual del proyecto a realizar}
\label{sec:descripcion}

El Radio Observatorio de Jicamarca es una instalación de investigación científica gestionada por el Instituto Geofísico del Perú (IGP), reconocida mundialmente por sus estudios sobre la ionósfera. El sistema cambio de apunte atuomático (ABS) fue diseñado originalmente para facilitar el cambio de apuntamiento de las antenas de manera manual o remota a través de Ethernet, asegurando la precisión requerida para estos experimentos. Sin embargo, a medida que los sistemas de monitoreo y gestión se han vuelto más sofisticados, ha surgido la necesidad de actualizar el sistema para mejorar la trazabilidad, mantenimiento y diagnóstico de fallos.

El proyecto consiste en la mejora del sistema ABS del Radio Observatorio de Jicamarca, un sistema crítico para el control de antenas utilizadas en experimentos avanzados sobre la ionósfera terrestre. Esta mejora tiene como objetivo modernizar el sistema integrando tecnologías de Internet de las Cosas (IoT) para permitir el monitoreo, control y gestión remota de las operaciones, ampliando sus capacidades actuales y optimizando la eficiencia en el manejo de datos.

Actualmente, el sistema permite el control remoto del apuntamiento de antenas, pero carece de capacidades avanzadas para el registro de datos, diagnóstico proactivo y monitoreo ambiental, lo que puede comprometer la confiabilidad y disponibilidad del sistema. La propuesta es implementar una solución IoT que no solo mantenga las funciones básicas del sistema, sino que también integre:

\begin{itemize}
	\item Monitoreo de variables ambientales como la temperatura para detectar condiciones adversas.
	\item Registro histórico de fallas y mantenimiento, mejorando la trazabilidad y análisis de fallos.
	\item Gestión de datos en tiempo real mediante una interfaz web multiplataforma.
	\item Desarrollo de shield personalizada para el monitoreo de temperatura.
\end{itemize}

El sistema propuesto incluye los siguientes componentes:
\begin{itemize}
    \item Microcontrolador TIVA TM4C1294 para control central.

	\item Shield personalizado para protección y expansión modular de sensores adicionales.

	\item Sensores ambientales, inicialmente de temperatura, con posibilidad de expansión.

	\item Servidor MQTT para comunicación eficiente y escalable.

	\item Base de datos local para almacenar y gestionar el historial de fallas y eventos.

	\item Aplicación web para acceso multiplataforma a datos y configuraciones.

	\item Seguridad en redes locales para garantizar acceso controlado.	
\end{itemize}

El diagrama en bloques del sistema se presenta en la Figura \ref{fig:diagBloques}, donde se destacan el módulo ABS,procesamiento de datos y comunicación. y visualización.

\begin{figure}[htpb]
	\centering 
	\includegraphics[width=.90\textwidth]{./Figuras/diagBloques.png}
	\caption{Diagrama en bloques del sistema.}
	\label{fig:diagBloques}
\end{figure}

\vspace{25px}


\section{2. Identificación y análisis de los interesados}
\label{sec:interesados}

\begin{itemize}
	\item Auspiciante: es riguroso y exigente con la rendición de gastos. Tener mucho cuidado con esto.
	\item Cliente: el Radio Observatorio de Jicamarca, interesado en el desarrollo del proyecto para implementarlo en las mejoras de los futuros experimentos a realizar.
\end{itemize}


\begin{table}[ht]
%\caption{Identificación de los interesados}
%\label{tab:interesados}
\begin{tabularx}{\linewidth}{@{}|l|X|X|l|@{}}
\hline
\rowcolor[HTML]{C0C0C0} 
Rol           & Nombre y Apellido & Organización 	& Puesto 	\\ \hline
Auspiciante   & Radio Observatorio de     &    -          	&     -   	\\ \hline
Cliente       & \clientename      &\empclientename	& -       	\\ \hline
Impulsor      &                   &              	&        	\\ \hline
Responsable   & \authorname       & FIUBA        	& Alumno 	\\ \hline
Colaboradores &                   &              	&        	\\ \hline
Orientador    & \supname	      & \pertesupname 	& Director del Trabajo Final \\ \hline
Equipo        & - \newline 
				-          &   -           	&  -      	\\ \hline
Opositores    &                   &              	&        	\\ \hline
Usuario final &  Personal encargado en el sistema ABS           &              	&        	\\ \hline
\end{tabularx}
\end{table}

\section{3. Propósito del proyecto}
\label{sec:proposito}

El propósito de este proyecto es desarrollar una actualización integral del Sistema de Apunte Automático (ABS) del Radio Observatorio de Jicamarca, incorporando tecnologías de Internet de las Cosas (IoT) para mejorar su capacidad de monitoreo, control y gestión remota. Esta mejora permitirá un registro más detallado de datos operativos, diagnóstico de fallas en tiempo real y una interfaz de usuario más accesible y eficiente, optimizando así el rendimiento y la disponibilidad del sistema para las exigentes investigaciones científicas que se realizan en esta instalación.

\section{4. Alcance del proyecto}
\label{sec:alcance}

El alcance de este proyecto incluye el diseño, desarrollo y despliegue de una solución IoT para mejorar el Sistema de Apunte Automático (ABS) del Radio Observatorio de Jicamarca. Específicamente, se contempla:

\begin{itemize}
	\item La integración de un sensor de temperatura como variable inicial para monitoreo ambiental, con posibilidad de expansión para otros sensores en futuras etapas.

	\item El diseño y fabricación de un shield electrónico personalizado para protección y expansión modular del sistema, asegurando aislamiento eléctrico y compatibilidad con el microcontrolador TIVA TM4C1294.

	\item La migración del protocolo de comunicación actual hacia MQTT para mejorar la eficiencia, escalabilidad y gestión de datos.

	\item El desarrollo de una aplicación web multiplataforma para monitoreo y gestión del sistema, con almacenamiento local de datos históricos en contenedores Docker para mayor portabilidad y facilidad de mantenimiento.

	\item La implementación de medidas básicas de seguridad en red local para restringir el acceso no autorizado.

	\item Pruebas de funcionamiento y validación del sistema en condiciones reales en el Radio Observatorio de Jicamarca.
\end{itemize}

El presente proyecto no incluye:

\begin{itemize}
	\item El desarrollo de aplicaciones móviles nativas para iOS o Android.

	\item La implementación de algoritmos de inteligencia artificial o modelos predictivos para análisis de datos.

	\item La integración con servicios en la nube para almacenamiento remoto o procesamiento avanzado de datos.

	\item El rediseño físico o estructural del sistema de antenas o sus componentes mecánicos.

	\item El suministro de infraestructura de red, como switches, routers o fibra óptica.
\end{itemize}

\section{5. Supuestos del proyecto}
\label{sec:supuestos}

Para el desarrollo del presente proyecto se supone que:
\begin{itemize}
	\item El sistema actual del Radio Observatorio de Jicamarca está en condiciones operativas adecuadas para integrar las mejoras propuestas.

	\item El microcontrolador TIVA TM4C1294 es compatible con el shield personalizado y los sensores adicionales que se implementarán.

	\item Los sensores de temperatura y otros dispositivos de monitoreo estarán disponibles y serán compatibles con el entorno de operación del sistema ABS.

	\item El equipo de trabajo contará con acceso a las instalaciones del observatorio para pruebas, instalación y validación del sistema.

	\item El protocolo MQTT es adecuado para la transmisión de datos en el entorno local del observatorio, cumpliendo con los requisitos de latencia y estabilidad.

	\item Se dispondrá del conocimiento técnico para diseñar e implementar las interfaces de comunicación, el shield personalizado y el sistema de almacenamiento de datos.

	\item Las condiciones de red local serán suficientemente estables para permitir una conectividad constante entre el microcontrolador y el servidor de datos.

	\item No se prevén cambios regulatorios significativos que afecten el uso de tecnologías IoT en el ámbito de las telecomunicaciones del observatorio.

	\item Los recursos económicos y materiales necesarios para el diseño, fabricación y pruebas del shield personalizado estarán disponibles a lo largo del proyecto.
\end{itemize}

\section{6. Requerimientos}
\label{sec:requerimientos}

\begin{enumerate}
	\item Requerimientos funcionales:
	\begin{enumerate}
		\item El sistema debe permitir el monitoreo continuo de la temperatura en tiempo real. (Prioridad alta)
		\item El sistema debe registrar y almacenar eventos críticos, fallas y cambios de apuntamiento en una base de datos local. (Prioridad alta)
		\item El sistema debe permitir el control remoto del apuntamiento de antenas a través de una interfaz web multiplataforma. (Prioridad alta)
		\item El sistema debe enviar notificaciones de fallas a dispositivos conectados en la red local. (Prioridad media)
		\item La interfaz web debe ser compatible con navegadores modernos y adaptarse a dispositivos móviles. (Prioridad media)
		\item El sistema debe permitir la descarga de registros históricos en formato CSV o similar para análisis externo. (Prioridad media)
	\end{enumerate}
	\item Requerimientos de Hardware:
	\begin{enumerate}
		\item Uso de un microcontrolador TIVA TM4C1294 como núcleo del sistema. (Prioridad alta)
		\item Desarrollo de un shield electrónico personalizado para protección y expansión de sensores. (Prioridad alta)
		\item Inclusión de al menos un sensor de temperatura compatible con el microcontrolador. (Prioridad alta)
		\item  Fuente de alimentación y conversión de medios (fibra a Ethernet) para garantizar conectividad estable. (Prioridad alta)
	\end{enumerate}
	\item Requerimientos de Comunicación:
	\begin{enumerate}
		\item Implementación del protocolo MQTT para intercambio de datos entre el microcontrolador y los dispositivos conectados. (Prioridad alta)
		\item Configuración de un servidor MQTT local para la gestión de mensajes y datos en red cerrada. (Prioridad alta)
		\item Implementación de medidas básicas de seguridad para restringir el acceso no autorizado. (Prioridad alta)
	\end{enumerate}
	\item Requerimientos de Software:
	\begin{enumerate}
		\item Desarrollo de una aplicación web para gestión y visualización de datos. (Prioridad alta)
		\item Uso de contenedores Docker para facilitar la portabilidad y mantenimiento del servidor de datos. (Prioridad media)
		\item Compatibilidad con bases de datos locales para almacenamiento de históricos. (Prioridad alta)
	\end{enumerate}
	\item Requerimientos regulatorios y normativos:
	\begin{enumerate}
		\item Cumplimiento con las regulaciones locales de telecomunicaciones para transmisión de datos. (Prioridad alta)
		\item Asegurar que todos los componentes electrónicos cumplan con normas de seguridad eléctrica. (Prioridad alta)
		\item Cumplimiento con las normativas de protección de datos personales y privacidad en la red local. (Prioridad alta)
	\end{enumerate}
	\item Requerimientos opcionales:
	\begin{enumerate}
		\item Posibilidad de integración futura con algoritmos de inteligencia artificial para análisis predictivo. (Prioridad baja)
		\item Compatibilidad para agregar sensores adicionales, como humedad, presión o vibración. (Prioridad baja)
		\item Integración con servicios en la nube para acceso remoto en futuras etapas. (Prioridad baja)
	\end{enumerate}
\end{enumerate}

\section{7. Historias de usuarios (\textit{Product backlog})}
\label{sec:backlog}

\begin{enumerate}
	\item Como operador del sistema, quiero poder monitorear la temperatura en tiempo real para detectar condiciones adversas y evitar daños en los equipos.
	
	\textit{Story points}: 8 (complejidad: 3, dificultad: 2, incertidumbre: 3)
	
	\item Como ingeniero electrónico, quiero poder registrar y visualizar el historial de fallas para diagnosticar problemas rápidamente.
	
	\textit{Story points}: 10 (complejidad: 4, dificultad: 3, incertidumbre: 3)
	
	\item Como operador, quiero poder descargar los registros históricos para análisis detallado fuera de línea.
	
	\textit{Story points}: 7 (complejidad: 2, dificultad: 2, incertidumbre: 3)
	
	\item Como usuario del sistema, quiero acceder a la interfaz web desde cualquier dispositivo para monitorear y controlar las funciones del sistema de forma conveniente.
	
	\textit{Story points}: 9 (complejidad: 3, dificultad: 3, incertidumbre: 3)
	
\end{enumerate}

\section{8. Entregables principales del proyecto}
\label{sec:entregables}

Los entregables del proyecto son:

\begin{itemize}
	\item Manual de usuario.
	\item Diagrama de circuitos esquemáticos.
	\item Código fuente del firmware.
	\item Diagrama de instalación.
	\item Aplicación Web Multiplataforma.
	\item Base de datos local.
	\item Memoria del trabajo final.
\end{itemize}


\section{9. Desglose del trabajo en tareas}
\label{sec:wbs}

\begin{enumerate}
\item Diseño y Planificación del Sistema (51 hs)
	\begin{enumerate}
	\item Análisis de requisitos técnicos y funcionales. (10 hs)
	\item Diseño del shield personalizado. (15 hs) 
	\item Definición de arquitectura de hardware y software. (12 hs)
	\item Selección de sensores y componentes adicionales. (6 hs)
	\item Documentación inicial del proyecto. (8 hs)
	\end{enumerate}
\item Diseño general del proyecto (35 hs)
	\begin{enumerate}
	\item Realización de diagrama de bloques. (4 hs)
	\item Realización de diseño de la arquitectura del sistema. (4 hs)
	\item Obtener los componentes para el prototipo de pruebas. (12 hs)
	\item Diagrama de flujo del programa. (15 hs)
	\end{enumerate}
\item Desarrollo del Hardware (56 hs)
	\begin{enumerate}
	\item Diseño del esquemático del shield personalizado. (12 hs)
	\item Diseño y fabricación del PCB del shield. (35 hs)
	\item Montaje y soldadura de componentes. (9 hs)
	\end{enumerate}
\item Desarrollo del Firmware (127 hs)
	\begin{enumerate}
	\item Programación del microcontrolador TIVA TM4C1294. (25 hs)
	\item Implementación del protocolo MQTT. (22 hs)
	\item Desarrollo del código para manejo de sensores. (30 hs)
	\item Pruebas unitarias y depuración. (40 hs)
	\end{enumerate}
\item Desarrollo de Aplicación Web (180 hs)
	\begin{enumerate}
	\item Diseño de la interfaz de usuario. (20 hs)
	\item Programación del backend para gestión de datos. (35 hs)
	\item Integración con base de datos local. (20 hs)
	\item Programación del frontend para desarrollo de la aplicación Web. (40 hs)
	\item Integración entre el backend y frontend. (40 hs)
	\item Pruebas de funcionalidad y ajustes finales (25 hs)
	\end{enumerate}
\item Pruebas y Validación del Sistema (110 hs)
	\begin{enumerate}
	\item Pruebas de integración hardware-software. (40 hs)
	\item Pruebas de rendimiento del sistema en condiciones reales. (40 hs)
	\item Validación de comunicación y seguridad de datos. (30 hs)
	\end{enumerate}
\item Documentación y Entrega Final (20 hs)
	\begin{enumerate}
	\item Elaboración del manual de usuario. (6 hs)
	\item Documentación técnica del shield y firmware. (8 hs)
	\item Generación del informe final y memoria del proyecto. (6 hs)
	\end{enumerate}
\item Documentación y Entrega Final (29 hs)
	\begin{enumerate}
	\item Elaboración del manual de usuario. (6 hs)
	\item Documentación técnica del shield y firmware. (8 hs)
	\item Elaboración del manual para el desarrollador. (15 hs)
	\end{enumerate}
\item Presentación del trabajo (55 hs)
	\begin{enumerate}
	\item Elaborar la memoria técnica del trabajo final. (40 hs)
	\item Elaborar la presentación del trabajo final. (15 hs)
	\end{enumerate}
\end{enumerate}

Cantidad total de horas: 663 hs.
%\vspace{60px}

%\vspace{230px}
\newpage
\section{10. Diagrama de Activity On Node}
\label{sec:AoN}

\begin{figure}[htpb]
	\centering 
	\includegraphics[width=1.0\textwidth]{./Figuras/AoN.png}
	\caption{Diagrama en Activity on Node.}
	\label{fig:AoN}
\end{figure}
\newpage
\section{11. Diagrama de Gantt}
\label{sec:gantt}

Para el diagrama de Gantt se consideró una dedicación parcial promedio de 3 hs durante todos los días hábiles.

\begin{figure}[htpb]
	\centering 
	\includegraphics[width=0.82\textwidth]{./Figuras/gantt_1.png}
	\caption{Diagrama en gantt desarrollado en Gantt Project}
	\label{fig:Gannt1}
\end{figure}


\begin{landscape}
\begin{figure}[htpb]
\centering 
\includegraphics[height=1\textheight]{./Figuras/gantt2.png}
\caption{Diagrama en gantt desarrollado en Gantt Project} 
\label{fig:Gantt2}
\end{figure}

\end{landscape}


\section{12. Presupuesto detallado del proyecto}
\label{sec:presupuesto}

La moneda utilizada en el presupuesto es el dólar.

\begin{table}[htpb]
\centering
\begin{tabularx}{\linewidth}{@{}|X|c|r|r|@{}}
\hline
\rowcolor[HTML]{C0C0C0} 
\multicolumn{4}{|c|}{\cellcolor[HTML]{C0C0C0}COSTOS DIRECTOS} \\ \hline
\rowcolor[HTML]{C0C0C0} 
Descripción &
  \multicolumn{1}{c|}{\cellcolor[HTML]{C0C0C0}Cantidad} &
  \multicolumn{1}{c|}{\cellcolor[HTML]{C0C0C0}Valor unitario} &
  \multicolumn{1}{c|}{\cellcolor[HTML]{C0C0C0}Valor total} \\ \hline
Mano de obra  &
  \multicolumn{1}{c|}{663} & 
  \multicolumn{1}{c|}{\$ 4.00} &
  \multicolumn{1}{c|}{\$ 2652} \\ \hline
Sensor de temperatura &
  \multicolumn{1}{c|}{3} &
  \multicolumn{1}{c|}{\$ 5.00} &
  \multicolumn{1}{c|}{\$ 15.00} \\ \hline
Otros componentes &
\multicolumn{1}{c|}{1} &
   \multicolumn{1}{c|}{\$ 30.00} &
   \multicolumn{1}{c|}{\$ 30.00} \\ \hline
\multicolumn{3}{|c|}{SUBTOTAL} &
  \multicolumn{1}{c|}{\$ 2697} \\ \hline
\rowcolor[HTML]{C0C0C0} 
\multicolumn{4}{|c|}{\cellcolor[HTML]{C0C0C0}COSTOS INDIRECTOS} \\ \hline
\rowcolor[HTML]{C0C0C0} 
Descripción &
  \multicolumn{1}{c|}{\cellcolor[HTML]{C0C0C0}Cantidad} &
  \multicolumn{1}{c|}{\cellcolor[HTML]{C0C0C0}Valor unitario} &
  \multicolumn{1}{c|}{\cellcolor[HTML]{C0C0C0}Valor total} \\ \hline
30\% de costos dirctos  &
\multicolumn{1}{c|}{1} &
\multicolumn{1}{c|}{\$ 809.1} &
\multicolumn{1}{c|}{\$ 809.1} \\ \hline

\multicolumn{3}{|c|}{SUBTOTAL} &
\multicolumn{1}{c|}{\$ 809.1} \\ \hline
\rowcolor[HTML]{C0C0C0}
\multicolumn{3}{|c|}{TOTAL} &
\multicolumn{1}{c|}{\$ 3506.1} \\ \hline
\end{tabularx}%
\end{table}


\section{13. Gestión de riesgos}
\label{sec:riesgos}

 
Riesgo 1: fallo en el microcontrolador TIVA TM4C1294
\begin{itemize}
	\item Severidad (S): 9 \\ 
	El sistema depende completamente del microcontrolador para el control de antenas y gestión de datos, por lo que su fallo implica una pérdida crítica de funcionalidad.
	\item Probabilidad de ocurrencia (O): 3 \\
	Los microcontroladores TIVA son altamente confiables, pero pueden fallar por sobrecargas, errores de diseño del shield o daños físicos.
\end{itemize}   

Riesgo 2: interferencia en la comunicación MQTT
\begin{itemize}
	\item Severidad (S): 8 \\ 
	Una falla en la comunicación puede dejar el sistema sin datos críticos y dificultar el monitoreo remoto.
	\item Probabilidad de ocurrencia (O): 3 \\
	Las redes locales pueden experimentar interferencias o congestión, especialmente en ambientes con alta demanda de ancho de banda.
\end{itemize}   

Riesgo 3: daño físico al shield personalizado
\begin{itemize}
	\item Severidad (S): 7 \\ 
	Un daño en el shield puede comprometer la protección eléctrica del sistema y generar fallos críticos.
	\item Probabilidad de ocurrencia (O): 4 \\
	El proceso de soldadura y montaje conlleva riesgos si no se toman las precauciones adecuadas.
\end{itemize}   

Riesgo 4: fallo en los sensores de temperatura
\begin{itemize}
	\item Severidad (S): 6 \\ 
	Los datos de temperatura son importantes para evitar sobrecalentamientos y daños en los equipos.
	\item Probabilidad de ocurrencia (O): 6 \\
	Los sensores son componentes relativamente sensibles que pueden dañarse por picos de voltaje o condiciones ambientales extremas.
\end{itemize}   

Riesgo 5: problemas de seguridad en la red local
\begin{itemize}
	\item Severidad (S): 10 \\ 
	Una brecha de seguridad puede exponer información sensible y comprometer la estabilidad del sistema.
	\item Probabilidad de ocurrencia (O): 5 \\
	Aunque se planea implementar medidas básicas de seguridad, siempre existe el riesgo de ataques externos o internos.
\end{itemize}   



b) Tabla de gestión de riesgos: (El RPN se calcula como RPN=SxO)

\begin{table}[htpb]
\centering
\begin{tabularx}{\linewidth}{@{}|X|c|c|c|c|c|c|@{}}
\hline
\rowcolor[HTML]{C0C0C0} 
Riesgo & S & O & RPN & S* & O* & RPN* \\ \hline
     1 & 9 & 3 &  27 &    &    &      \\ \hline
     2 & 8 & 3 &  24 &    &    &      \\ \hline
     3 & 7 & 4 &  28 &    &    &      \\ \hline
     4 & 6 & 6 &  36 & 5  & 2  & 10   \\ \hline
     5 & 10 & 5 & 50 & 5  & 5  & 25   \\ \hline
\end{tabularx}%
\end{table}

Criterio adoptado: 

Se tomarán medidas de mitigación en los riesgos cuyos números de RPN sean mayores a 30

Nota: los valores marcados con (*) en la tabla corresponden luego de haber aplicado la mitigación.

c) Plan de mitigación de los riesgos que originalmente excedían el RPN máximo establecido:
 
Riesgo 4: fallo en los sensores de temperatura
  \begin{itemize}
  	\item Plan de mitigación: Utilizar sensores de mayor calidad con certificación industrial para mayor precisión y durabilidad.
	\item Severidad (S): 5 \\ 
	La severidad es moderada, si el sistema utiliza unos sensores certificados.
	\item Probabilidad de ocurrencia (O): 2 \\
	La probabilidad de ocurrencia es baja, porque se usará sensores que se cuentan en el inventario.
	\end{itemize}
\newpage
Riesgo 5: problemas de seguridad en la red local
\begin{itemize}
	\item Plan de mitigación: Implementar cifrado de datos en tránsito usando TLS/SSL en las comunicaciones MQTT.
	\item Severidad (S): 5 \\ 
	La severidad es moderada, si el sistema logra cifrar la data adquirida.
	\item Probabilidad de ocurrencia (O): 5 \\
	Es poco probable, pero se utilizará servicios ya utilizados en el radio observatorio.
\end{itemize}   

\section{14. Gestión de la calidad}
\label{sec:calidad}

\begin{itemize} 

\item Req 1.1: El sistema debe permitir el monitoreo continuo de la temperatura en tiempo real.

\begin{itemize}
	\item Verificación: Se conectará el sensor de temperatura al microcontrolador y se registrará la lectura en intervalos de 1 segundo. Se utilizará un sensor patrón para comparar los valores medidos. Se revisará el código para asegurar lectura continua y estabilidad.
	\item 	Validación: Se presentará al cliente la interfaz web mostrando la temperatura en tiempo real, utilizando una pistola termómetro para comparar el valor en vivo con una referencia externa.
\end{itemize}

\item Req 1.2: El sistema debe registrar y almacenar eventos críticos, fallas y cambios de apuntamiento en una base de datos local.

\begin{itemize}
	\item Verificación: Se forzarán eventos simulados (fallas de conexión, cambio de apuntamiento) y se verificará que los datos se registren correctamente en la base de datos.

	\item Validación: El cliente accederá a la interfaz y consultará eventos registrados; validará que los eventos coincidan con las acciones que se ejecutaron durante la prueba.

\end{itemize}

\item Req 1.3: El sistema debe permitir el control remoto del apuntamiento de antenas a través de una interfaz web multiplataforma.

\begin{itemize}
	\item Verificación: Se probará el control remoto desde distintos dispositivos en la red local. Se inspeccionará el código del backend y las señales eléctricas enviadas a los relés de apuntamiento.

	\item Validación: El cliente controlará el sistema desde una PC y un smartphone y observará que el cambio de apuntamiento se refleja correctamente en el hardware.
\end{itemize}

\item Req 1.4: El sistema debe enviar notificaciones de fallas a dispositivos conectados en la red local.
\begin{itemize}
	\item Verificación: Se simularán condiciones de falla (desconexión de sensores, sobrecarga) y se verificará que los mensajes MQTT lleguen a los suscriptores configurados.

	\item Validación: El cliente recibirá notificaciones en tiempo real en su panel de usuario y confirmará que reflejan el estado del sistema.
\end{itemize}

\item Req 1.5: La interfaz web debe permitir el acceso desde computadoras, tablets y smartphones.
\begin{itemize}
	\item Verificación: Se probará la interfaz en múltiples dispositivos y navegadores (Chrome, Firefox, Safari) y se verificará la adaptación responsiva del diseño.
	
	\item Validación: El cliente probará la interfaz desde sus propios dispositivos para confirmar compatibilidad y usabilidad.
\end{itemize}

\item Req 1.6: El sistema debe permitir la descarga de registros históricos en formato CSV.
\begin{itemize}
	\item Verificación: Se generará un archivo CSV desde la interfaz, se verificará que contiene los campos esperados y se abrirá en Excel para comprobar estructura.

	\item Validación: El cliente descargará el archivo y lo usará para realizar un análisis básico de fallas y eventos. Confirmará que los datos están completos y claros.
\end{itemize}
\item Req 2.2: El shield personalizado debe proteger y facilitar la conexión de sensores.
\begin{itemize}
	\item Verificación: Se realizará una revisión de diseño del esquemático y se probarán las entradas con distintas tensiones para confirmar aislamiento.

	\item Validación: Se mostrará al cliente el funcionamiento estable del sistema durante las pruebas de campo, confirmando que los sensores funcionan correctamente sin daño al microcontrolador.
\end{itemize}

\item Req 3.1: Implementación del protocolo MQTT para intercambio de datos entre el microcontrolador y los dispositivos conectados.

\begin{itemize}
	\item Verificación: Se revisará el código fuente para confirmar el uso del cliente MQTT, y se capturarán paquetes con Wireshark para comprobar el protocolo.
	
	\item Validación: El cliente recibirá datos vía MQTT en una aplicación externa compatible (ej. MQTT Explorer), validando que la comunicación es funcional.
\end{itemize}

\item Req 4.1: Desarrollo de una aplicación web para gestión y visualización de datos.
\begin{itemize}
	\item Verificación: Se revisará el código fuente de la aplicación web (HTML, CSS, JavaScript, backend) para comprobar que incluye todos los módulos requeridos.
	
	\item Validación: El cliente utilizará la aplicación web en diferentes dispositivos (PC, tablet, smartphone) y verificará que puede acceder a las funcionalidades críticas del sistema.
	\end{itemize}
	
\item Req 4.2: Uso de contenedores Docker para facilitar la portabilidad y mantenimiento del servidor de datos.
\begin{itemize}
	\item Verificación:Se examinará el archivo Dockerfile y los docker-compose.yml utilizados, asegurando que el sistema se construya correctamente en un entorno limpio. 

	\item Validación: Se considerará validado si el cliente puede ejecutar el sistema sin complicaciones, accede a la interfaz y confirma que todos los servicios están operativos tras la instalación.
\end{itemize}

\end{itemize}

\section{15. Procesos de cierre}    
\label{sec:cierre}

Se contemplan las siguiente actividades una vez finalizado al cierre del proyecto.

\begin{itemize}
	\item Análisis de seguimiento del Plan de Proyecto original
	\begin{itemize}
		\item Responsable: William’s Limonchi
		\item Actividad: 
		\begin{itemize}
			\item Se comparará el cronograma real del proyecto con el cronograma inicial planificado, utilizando la documentación de seguimiento y los registros de avance por tareas.
			\item Se verificará el cumplimiento de los requerimientos definidos mediante una lista de cotejo cruzada con los entregables.
		\end{itemize}
	\end{itemize}
	\item Identificación de procedimientos útiles, problemas que surgieron y cómo se solucionaron:
	\begin{itemize}
		\item Responsable: William’s Limonchi
		\item Actividad: 
		\begin{itemize}
			\item Se analizará el uso de las diferentes herramientas utilizadas para evaluar el
			impacto en el proyecto.
			\item Se presentarán los inconvenientes detectados con la solución respectiva, a fin de
			evitar que vuelvan a ocurrir.
		\end{itemize}
	\end{itemize}
	\item Indicar quién organizará el acto de agradecimiento a todos los interesados.
	\begin{itemize}
		\item Responsable: William’s Limonchi
		\item Actividad: 
		\begin{itemize}
			\item Al finalizar el proyecto y presentar la defensa pública, se procederá a agradecer
			a todas las personas que han participado en el desarrollo del proyecto, jurados,
			compañeros, docentes y autoridades de la carrera de especialización.
		\end{itemize}
	\end{itemize}
\end{itemize}



\end{document}